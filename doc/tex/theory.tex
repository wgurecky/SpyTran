%-----------------------------------------------------------------------------%
This work demonstrates the application of the discontinuities galerkin finite elements (DGFE)
to the radiation transport equation in 1D.  This work can be extended to two or
three dimensions however, the core concepts of the DG method are readily conveyed by the 1D case.

\section{Background}

We begin by considering the time-independent, single dimensional, non-multiplying fixed source neutron transport equation given by equation \ref{1dtr}.  This equation is continuous in energy, space and angle.

\begin{eqnarray}
& \left[ \Omega \cdot \nabla + \Sigma_t -
\int_{E'} dE' \int_{\mathbf\Omega'} d \mathbf\Omega'
\Sigma_s \left(x, E' \rightarrow E , \mathbf\Omega' \cdot \mathbf\Omega \right) \right] 
\psi(x, \mathbf\Omega, E)  \nonumber \\
& =
Q(x, \mathbf\Omega, E)
\label{1dtr}
\end{eqnarray}

$\Sigma_t$ is the total macroscopic cross section.  $Q$ is an arbitrary non-fission volumetric neutron source. 
$\Sigma_s \left(x, E' \rightarrow E , \mathbf\Omega' \cdot \mathbf\Omega \right)$ is sometimes called the differential scattering cross section.  This describes the probability per unit length of scattering from $E'$ to the infinitesimal energy window $dE$ about $E$ in coincidence with scattering from angle $\mathbf\Omega'$ into the infinitesimal solid angle cone $d\mathbf\Omega$ about $\mathbf\Omega$.  Note the differential cross section depends on the dot product between the incomming and outgoing scattering angles.  

Equation \ref{trop} is referred to as the streaming-collision operator.  This operates on the angular dependent flux, $\psi$, and concisely conveys the redistribution of neutrons in space due to non-fission neutron interactions with matter.
\begin{equation}
H = \left[ \Omega \cdot \nabla + \Sigma_t 
 \right]
\label{trop}
\end{equation}

The scattering operator, $S$ is given by equation \ref{scatter_op} and governs the redistribution of neutrons in angle and energy due to scattering interactions.
\begin{equation}
S = \left[ 
\int_{E'} dE' \int_{\mathbf\Omega'} d \mathbf\Omega'
\Sigma_s \left(x, E' \rightarrow E , \mathbf\Omega' \cdot \mathbf\Omega \right)
 \right]
\label{scatter_op}
\end{equation}

We can concisely write the transport equation in operator notation in equation \ref{trop_not}.

\begin{equation}
H\psi = S\psi + Q
\label{trop_not}
\end{equation}
$H-S$ is sometimes defined to be the transport operator.

The next step is to apply the discrete ordinate approximation in angle and the multigroup approximation in energy.  In this case, we restrict the neutrons to travel in a finite (and typically small) number, $G$, of energy groups.  Neutrons in group $g$ are said to have some energy on the interval $(E_g, E_{g-1}]$.  The continuous energy integral in \ref{1dtr} can
be replaced by the sum shown in equation \ref{mgrp}.

\begin{equation}
\int_0^\infty dE' \approx \sum_{g'=1}^G \int_{E_{g'-1}}^{E_{g'}} dE'
\label{mgrp}
\end{equation}

Before applying the multigroup approximation to \ref{1dtr}, it is advantageous to replace the differential scattering operator with a spherical harmonic series expansion. This is done to eliminate the angular dependence of the cross sections.  Most numerical treatments of neutron transport do not use angular dependant cross sections and instead capture the angular redistribution action of the scattering operator through, typically, a Legendre expansion \cite{Lewis}, (pg. 36).

\begin{eqnarray}
& S\psi \approx \nonumber \\
& \sum_{l=0}^N (2l+1) \int dE' \Sigma_{sl}(x, E'\rightarrow E)
  \int d\mathbf\Omega' P_l(\mathbf\Omega \cdot \mathbf{\Omega'}) \psi
\label{sexp}
\end{eqnarray}
Where $l$ is the Legendre polynomial ($P_l$) order and $N$ is the maximum number of terms to retain in the expansion.  We see that the angular dependence has been eliminated from the scattering cross section $\Sigma_s$.
The Legendre addition theorem provides the identity:
\begin{equation}
P_l(\mathbf\Omega \cdot \mathbf{\Omega'}) = \frac{1}{2l+1}
\sum_{m=-l}^{l} Y^{*}_{lm}(\mathbf\Omega) Y_{lm}(\mathbf\Omega')
\label{leg_add}
\end{equation}
Where $Y_{lm}$ is the spherical harmonic of degree $l$ and order $m$.  $*$ represents the complex conjugate.  Applying \ref{leg_add} to \ref{sexp} produces \ref{sexp2}.
\begin{eqnarray}
& S\psi \approx \nonumber \\
& \sum_{l=0}^N \int dE' \Sigma_{sl}(x, E'\rightarrow E) \sum_{m=-l}^{l} Y^{*}_{lm}(\mathbf\Omega)
  \int d\mathbf\Omega' Y_{lm}(\mathbf\Omega') \psi
\label{sexp2}
\end{eqnarray}
The inner product of the spherical harmonics with the angular dependent flux over angle $\mathbf\Omega'$ is represented by the scalar values, $\phi_l^m(x, E)$, \ref{sph_coeffs}:
\begin{equation}
\int d\mathbf\Omega' Y_{lm}(\mathbf\Omega') \psi(x, \mathbf\Omega', E) = \phi_l^m(x, E)
\label{sph_coeffs}
\end{equation}
We can rewrite \ref{sexp2} as shown in \ref{sexp3}.
\begin{eqnarray}
& S\psi \approx \nonumber \\
& \sum_{l=0}^N \int dE' \Sigma_{sl}(x, E'\rightarrow E) \sum_{m=-l}^{l} Y^{*}_{lm}(\mathbf\Omega) \phi_l^m(x, E)
\label{sexp3}
\end{eqnarray}

In one dimension, we are only concerned with the direction cosine with respect to the spatial coordiante axis, $\mu=\mathbf\Omega \cdot \mathbf x=cos(\theta)$, due to symmetry.  Equation \ref{sexp3} reduces to equation \ref{sexp4}.
\begin{eqnarray}
& S\psi \approx \nonumber \\
& \sum_{l=0}^N (2l+1) P_l(\mu) \int dE' \Sigma_{sl}(x, E'\rightarrow E) \phi_l(x, E)
\label{sexp4}
\end{eqnarray}
Where $\phi_l$ is the $l^{th}$ Legendre moment of the flux given by equation \ref{leg_mom_flx}.
\begin{equation}
\phi_l(x,E) = \frac{1}{2} \int_{-1}^1 d\mu' P_l(\mu') \psi(x, \mu', E)
\label{leg_mom_flx}
\end{equation}

Now we can rewrite the one dimensional transport equation (\ref{1dtr}) as shown in equation \ref{1dtr2}.
\begin{eqnarray}
& \mu \frac{\partial}{\partial x} \psi + \Sigma_t \psi -
\sum_{l=0}^N (2l+1) P_l(\mu) \int dE' \Sigma_{sl}(x, E'\rightarrow E) \phi_l(x, E)  
 \nonumber \\
& =
Q(x, \mu, E)
\label{1dtr2}
\end{eqnarray}

Finally we apply the multigroup approximation which yields the desired \emph{energy discretized} non-multiplying 1D transport equation \ref{enrg_1dtr} for group $g$.
\begin{eqnarray}
& \mu \frac{\partial}{\partial x} \int_g \psi dE + \int_g \Sigma_t \psi dE -
\sum_{l=0}^N (2l+1) P_l(\mu) 
\sum_{g'=1}^G \int_g dE \int_{g'} dE' \Sigma_{sl}(x, E'\rightarrow E) \phi_l(x, E)  
 \nonumber \\
& =
\int_g Q(x, \mu, E) dE
\label{enrg_1dtr}
\end{eqnarray}
Where the integral $\int_g dE$ is shorthand for:
\begin{equation}
\int_g dE = \int_{E_{g-1}}^{E_g} dE
\end{equation}
Equation \ref{enrg_1dtr} can be written more concisely as:
\begin{eqnarray}
& \mu \frac{\partial}{\partial x} \psi_g(x,\mu) + \Sigma_{tg} \psi_g(x,\mu) -
\sum_{l=0}^N (2l+1) P_l(\mu) 
\sum_{g'=1}^G \Sigma_{lgg'} \phi_{lg'}(x)
 \nonumber \\
& =
Q_g(x, \mu)
\label{enrg_1dtr2}
\end{eqnarray}
Where $\Sigma_{lgg'}$ is the scattering kernel and is derived in E.E Lewis's text \cite{Lewis}.
The group total cross section is given by:
\begin{equation}
\Sigma_{tg} = \frac{\int_g \Sigma_t(E)\phi (E) dE} {\int_g \phi (E) dE}
\end{equation}
Unfortunately these group cross sections depend on already knowing the scalar flux as a function of energy.  This complication is the subject of an enormous quantity of literature on neutron self-shielded cross section generation.  The reader can use the program NJOY to generate approximate self shielded cross sections for a variety of assumed background spectra, if desired.

There is still an angular and spatial dependence in the energy discretized equation \ref{enrg_1dtr2}, therefore we apply the discrete ordinates approximation to simplify the equation further and prepare for the application of the DGFE spatial discretization technique.

The discrete ordinates approximation requires \ref{enrg_1dtr2} to hold for only a few specified directions, $\mu^n \in [1, -1]$ where $\mu$ is the cosine of the ordinate direction with the $x$ axis.  Additionally, we apply a quadrature approximation to the integral term.

This deceivingly simple assumption leads to the angle and energy discretized transport equation \ref{trf}.
\begin{equation}
    \mu^n \frac{\partial}{\partial x} \psi^n_{g}(x) + \Sigma_{tg} \psi^n_g(x) =
\sum_{l=0}^L (2l+1) P_l(\mu^n) 
\sum_{g'\neq g}^G \Sigma_{lgg'} \phi_{lg'}(x)
 + Q(x)
\label{trf}
\end{equation}
The superscript $n$ denotes a fixed ordinate index rather than an exponent.
$\phi_{lg'}(x)$ are the Legendre moments of the flux computed by a simple quadrature rule given by equation \ref{phi2}.
\begin{equation}
\phi_{lg'}(x) = \frac{1}{2}\sum_{n=1}^N w_n P_l(\mu^n)\psi^n_{g'}(x)
\label{phi2}
\end{equation}
The weights $w_n$ are typically computed using a level-symmetric or Gauss-Legendre quadrature rule and can be found in \cite{Lewis}.  $N$ is the number of ordinate directions to track.

Equation \ref{trf} be rewritten simply as:
\begin{equation}
    \mu^n \frac{\partial}{\partial x} \psi^n_{g}(x) + \Sigma_{tg} \psi^n_g(x) = S(x) + Q(x)
\label{g_mu_treq}
\end{equation}
Where $S(x)$ is the scattering source term from all other energy groups $g' \neq g$ and from all other angles $\mu_n' \ne \mu_n$.  We will take the volumetric source , $Q$ to be zero in the following work.

Though this transport equation holds only for the extreemly limited case of a steady-state non multiplying medium in one-dimension, it provides a useful starting point to investigate the spatial discretization of the neutron flux.
As previously stated, we focus on the DGFE approach here.
